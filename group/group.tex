\documentclass[10pt]{article}

\usepackage{times}
\usepackage{amsmath, amsthm, amssymb, amsfonts}

\theoremstyle{plain}% default
\newtheorem{theorem}{Theorem}[section]
\newtheorem{lemma}[theorem]{Lemma}
\newtheorem{proposition}[theorem]{Proposition}
\newtheorem*{corollary}{Corollary}

\theoremstyle{definition}
\newtheorem{definition}{Definition}[section]
\newtheorem{conjecture}{Conjecture}[section]
\newtheorem{example}{Example}[section]
\newtheorem{exercise}{Exercise}[section]

\theoremstyle{remark}
\newtheorem*{remark}{Remark}
\newtheorem*{note}{Note}
\newtheorem{case}{Case}

\begin{document}

\title{Review of group theory}

\maketitle

\section{Definitions in group theory}

\begin{definition}
	A magma is a set with a binary operation.
	A monoid is a unital associative magma.
	A group is a monoid with inverses.
\end{definition}

\begin{example}
	A group is a full subcategory of groupoids with a single object. The delooping functor from groups to groupoids is full and faithful.
\end{example}

\begin{definition}
	A group object internal to a category with finite products is an object with maps that have commuting diagrams for associativity, unitality and inverses. It is a presheaf functor onto the categories of groups whose underlying presheaf functor on sets is representable.
\end{definition}

\begin{example}
	A group object in group is an abelian group. This is an example of the Eckmann-Hilton argument.
\end{example}

\begin{definition}
	An automorphism is an endomorphism that is an isomorphism.
\end{definition}

\begin{definition}
	A group $G$ has a presentation for a coequaliser diagram from the free group on generators $FX$ to the free group on a set of relations $FR$.
	The diagram is $FR \rightrightarrows FX \rightarrow G$.
\end{definition}

\begin{definition}
	A finite group has an underlying finite set. It is a group object in the category of finite sets.
\end{definition}

\begin{definition}
	A subgroup is a subobject in the category of groups given by a monomorphism of groups from the subgroup to the full group.
\end{definition}

\begin{definition}
	A subgroup is normal if left conjugation by any element leaves the normal subgroup invariant. Equivalently, a normal subgroup are kernels of group homomorphisms. Loosely, they are congruence relations in the category of groups.
\end{definition}

\begin{definition}
	A quotient object of a congruence or internal equivalence relation of an object in a category is the coequaliser of the induced pair of morphisms that are made equivalent.
\end{definition}

\begin{definition}
	A quotient group is a quotient object in the category of groups.
\end{definition}

\begin{definition}
	A simple group is a group with exactly two quotient groups, the trivial group and the group itself.
\end{definition}

\begin{example}
	Consider the inclusion of a normal subgroup into a group, the quotient group is the set of cosets equipped with the group structure induced from the group $G$. It is precisely the cokernel of the inclusion if the inclusion is a (group homo)morphism of abelian groups.
\end{example}

\begin{definition}
	An action of a group in a category is a functor from the groupoid on the single object into a category.
	Enrichment over this category defines action objects.

	This can be encoded in the action groupoid fiber sequence in the category of groupoids.

	\begin{equation}
		X \rightarrow X // G \rightarrow \mathbf{B}G
	\end{equation}

	This fiber sequence may be thought of as the $\rho$-associated bundle to the $G$-universal principal bundle.
\end{definition}

\begin{example}
	A representation is a linear action using the general linear group. A circle action uses the circle group. An action on a set in itself if it exists, the set is a magma.
\end{example}

\begin{definition}
	A (left/right) coset object is the quotient of a group and a subgroup, this is a set of equivalence classes of elements of a group where two elements are equivalent if they differ by (left/right) multiplication with an element in the subgroup.

	The left coset object coequalises parallel morphisms from the product of the subgroup object and the full group object under group multiplication and projection.

	The right coset object coequalises parallel morphisms from the product of the full group object and the subgroup object (order matters here) under group multiplication and projection.
\end{definition}

\begin{definition}
	A normaliser subgroup of an underlying subset of a group is the subgroup with all elements such that there are two distinct elements in the underlying subset that commutes on the left and right with an element in this normaliser subgroup.
\end{definition}

\begin{definition}
	A centraliser subgroup is a normaliser subgroup with the stronger condition that these two distinct elements must be the same. So, all elements commute with the elements of a subset of the underlying set of the group.
\end{definition}

\begin{definition}
	The centre of a group is the subgroup with all underlying elements of the group that commutes with all elements of a group.
\end{definition}

\begin{definition}
	A stabiliser subgroup of an element in a full group is the set of all elements that leave the element fixed with the same group action. 
\end{definition}

\begin{definition}
	If there are two subgroups, they are conjugate subgroups if there exists an element such that a conjugation action by the group element takes one to the other.
\end{definition}

\begin{definition}
	The torsion subgroup is the subgroup of all elements with finite order whoser power is the netural element.

	A group is pure torsion if it coincides with its torsion subgroup.

	A group is torsion free if the torsion subgroup is the trivial group. The trivial group is the subgroup with only the neutral element.
\end{definition}

\begin{definition}
	A Sylow $p$-subgroup is a maximum $p$-torsion subgroup for a prime $p$ of a finite group $G$.
\end{definition}

\begin{definition}
	Given an action of a discrete group on a set, the set with the group acting on a fixed point is an orbit of the action, or the $G$-orbit through the point $x$.

	The set is a disjoint union of its orbits.

	The category of orbits of a group is the full subcategory of the category of sets with an action of the group $G$.
\end{definition}

\begin{example}
	An orbit of a cyclic subgroup of a permutation group is a permutation cycle.
\end{example}

\begin{definition}
	An adjoint action is an action by (left) conjugation.

	The conjugacy class of a element is the orbit of an element under the adjoint self-action of the group, It is the subset of all the elements obtained by conjugation from a group element with another group element.
\end{definition}

\begin{example}
	The conjugacy class of a neutral element is itself.

	An abelian group could be defined as a group where all conjugacy classes are singletons, one for each element of the group.

	The number of conjugacy classes refer to the number of its irreducible representations.

	The self duality of symmetric groups with general linear groups over the field of one element, so there is a correspondence between conjugacy classes and their irreducible representations.
\end{example}

\begin{example}
	A character of a linear representation is a well defined function that on the set of conjugacy classes of elements in the group. This is a neat definition, since such a function must be invariant under conjugation.
\end{example}

\begin{definition}
	A pullback is a limit of a cospan. It is the cone of a cospan that commutes. 
	A pushout is a colimit of a span. It is the cocone of a span that commutes. 
\end{definition}

\begin{definition}
	The direct product of groups is the cartesian product in the category of groups. If the group is abelian, and there is a finite number of factors, this is also the direct sum of groups.

	The free product of groups is the coproduct in the category of groups.

	The amalgamated free product of groups is the pushout or colimit in the category of groups. This is unique and universal by factoring a suitable homomorphism.
\end{definition}

\begin{definition}
	Let a group $G$ act on a group $\Gamma$ on the left by a group automorphism $p : G \rightarrow \mathrm{Aut}(\Gamma)$, the semidirect product group has an underlying set the cartesian product $\Gamma \times G$ with multiplication twisted by the group automorphism $p$ on the group element $\gamma$ in the group $\Gamma$, but not on $g$ in group $G$. The twisting acts on the group element $h$ in the group $G$:

	\begin{equation}
		(\delta, h)(\gamma, g)
		= (\delta p(h)(\gamma), hg))
	\end{equation}
\end{definition}

\begin{exercise}
	Find a better definition of the semidirect product.
\end{exercise}

\begin{definition}
	The Cayley graph (or quiver) of a group is a graph with the elements of the group $G$ as the vertices, and the edges labelled by elements of the set of generators $X$ of the group $G$ as subset of the underlying set of the group $G$. The metric induced from the graph distance is called the word metric with respect to the generators.
\end{definition}

\begin{example}[Cayley's theorem]
	Consider the group acting on itself by left multiplication with a permutation representation which can be drawn out on a Cayley graph. 
	This permutation representation maps the group to a subgroup of the symmetric group.

	This representation is faithful if the permutation representation is injective with a trivial kernel.
	
	Suppose a group element is in the kernel of the permutation representation, any group element is a left multiple of the identity element. Therefore, this group element must be the identity element since this is a permutation representation acting on the identity element, and the kernel of the permutation representation is trivial.

	Since the permutation representation has a trivial kernel, the image of the permutation representation must be isomorphic to the quotient group on the trivial kernel, which is the full group by the first isomorphism theorem. Therefore, the permutation representation is an isomorphism. Since this is an isomorphism, the group must be isomorphic to a subgroup of the symmetric group. 
\end{example}

\section{Class equation}

% Lagrange
% First, second, third, isomorphism theorems
% Sylow's Theorem
% Fundamental theorem of Ab groups
% Classification

Suppose $G$ is a group. Suppose $A$ is a $G$-set.

Each $G$-set $A$ has a canonical decomposition as coproducts of components. These are the orbits of the action. Define the representative element $a_x$ in each orbit $x$, there is a canonical isomorphism of $G$-set $A$ and the coproduct $\Sigma_{\mathrm{orbits \, in \, x}} G / \mathrm{Stab}(a_x)$.

Apply a forgetful functor into the cardinality of underlying sets, the cardinality of the $G$-set $A$ is the sum of the order of the group $G$ divided by the order of the stabiliser of the element $a_x$ under each orbit.

\begin{equation}
	|A| = \sum_{\mathrm{orbits \, x}}
	\dfrac{|G|}{|\mathrm{Stab}(a_x)|}
\end{equation}

This next form expresses the groupoid cardinality of the action groupoid of the group $G$ acting on the $G$-set, $A$.

\begin{equation}
	\dfrac{|A|}{|G|} = \sum_{\mathrm{orbits \, x}}
	\dfrac{1}{|\mathrm{Stab}(a_x)|}
\end{equation}

\begin{theorem}
	Show that a nontrivial group of prime power has a nontrivial centre.
\end{theorem}

\begin{proof}
	The orbit of an element if it belongs to the centre is consists of the element as a singleton if it belongs to the centre. Without loss of generality, consider elements that are not in the center.

	The cardinality of the orbit of an element divides the order of the group of prime power by the class equation. 
	
	The prime divides the cardinality of the orbit of a noncentral group element. 
	
	The class equation can be decomposed to calculate the order of the group as the sum of the order of the group center and the sum of the quotient of the order of group divided by the stabiliser of nontrivial orbits. 
	
	In both cases, the prime must divide the order of the group and the order of the centre. So the centre has more than one element and by definition, is nontrivial.
\end{proof}

\end{document}