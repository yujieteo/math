\documentclass[10pt]{article}

\usepackage{times}
\usepackage{amsmath, amsthm, amssymb, amsfonts}

\theoremstyle{plain}% default
\newtheorem{theorem}{Theorem}[section]
\newtheorem{lemma}[theorem]{Lemma}
\newtheorem{proposition}[theorem]{Proposition}
\newtheorem*{corollary}{Corollary}

\theoremstyle{definition}
\newtheorem{definition}{Definition}[section]
\newtheorem{conjecture}{Conjecture}[section]
\newtheorem{example}{Example}[section]
\newtheorem{exercise}{Exercise}[section]

\theoremstyle{remark}
\newtheorem*{remark}{Remark}
\newtheorem*{note}{Note}
\newtheorem{case}{Case}

\begin{document}

\title{An introduction to tensor products}

\maketitle

\begin{enumerate}
	\item The tensor product originally is a representing object for a multi-linear map. Classically, this is for modules over a ring.
	\item The tensor product functor that is part of the definition of monoidal categories is also a tensor product.
	\item It could also mean a tensor product over a monoid for a monoidal category with a right and left action. The tensor product is the over the monoid is the quotient of their tensor product in the monoidal category by this action.
\end{enumerate}

\begin{definition}
	A definition sketch for a coloured operad is a category with multiple inputs and one output for a collection of morphisms. Note that we use coloured operads to refer to multicategories, and coloured symmetric operads to symmetric multicategories.
\end{definition}

\begin{definition}
	The tensor product of a pair of objects in a coloured operad is an object with a universal multimorphism to the tensor product such that any multimorphism of this pair objects to another object factors uniquely through it.
\end{definition}

\begin{example}
	Supposed the coloured operad is the category of abelian groups, using multilinear maps as the multimorphisms. This gives the tensor product of abelian groups. The tensor product is equipped with a universal map from the cartesian product of a pair of abelian groups as a set to the tensor product such that the map is a linear (a group homomorphism in this case) in each argument separately.

	An explicit construction of the tensor product of abelian groups starts with the cartesian product in sets, generating a free abelian group of pairs from the underlying elements of the cartesian product, and quotienting by relations for the bilinearity on these pairs of group objects.
	
	The tensor product is neither a subobject or a quotient of the cartesian product in this case.
\end{example}

\begin{exercise}
	Use the above definition to define the tensor product of abelian semigroups.
\end{exercise}

\begin{theorem}
	A monoid in the category of abelian groups with the tensor product is a ring.
\end{theorem}

\begin{proof}
	The tensor product is bilinear and serves as the distributive law for the ring.
\end{proof}

\begin{example}
	The tensor product for the category of chain complex of modules of a commutative ring has components given by the coproduct of tensor products of modules whose degrees add to the degree of the component.
\end{example}

\begin{example}
	A category is closed if for any objects the collection of morphisms from objects in the category itself is an object. This object is denoted as the hom-object or the internal hom of the category. A coloured operad with unit can be constructed into a closed category.

	For any closed category if not monoidal has the underlying multicategory. Tensor products in this multicategory holds by the adjunction on the level of hom-objects between the tensor product and the object internal hom. This adjunction exists because internal homs exists in a closed category by definition.
	
	\begin{equation}
		\mathrm{hom}(A \otimes B, C)
		\cong
		\mathrm{hom}(A, \mathrm{hom}(B,C))
	\end{equation}

	Forgetting the monoidal structure of abelian groups, this internal hom is still there since the product of two group elements is a group element.
\end{example}

\begin{example}
	For a commutative ring $R$ with the multicategory $R\mathrm{Mod}$ of $R$-modules and $R$-multilinear maps over this commutative ring $R$. The tensor product of modules $A \otimes_R B$ can be constructed as quotient of the tensor product of abelian groups $A \otimes B$ with the equivalence relation of the action of the commutative ring $R$.

	\begin{equation}
		A \otimes R \otimes B \rightrightarrows A \otimes B
	\end{equation}

	which is the coequaliser of the two maps.

	If the commutative ring $R$ is a field, this is the tensor product of vector spaces. This also give rise to the tensor product of linear representations.

	This can even be generalised if the ring $R$ is not commutative, if $A$ is a right $R$-module and $B$ is a left $R$-module. More generally, if $R$ is a monoid in any monoidal category (a ring is a monoid in the category of abelian groups with its tensor product, the tensor product of a left and right $R$-module can be defined analogously)

	If $R$ is a commutative monoid in a symmetric monoidal category so that right and left modules coincide, the their tensor product of modules is again a $R$-module.

	Note that the tensor product in the category of abelian groups is not the tensor products of modules over any monoid in the cartesian monoidal category of sets.
\end{example}

\begin{definition}
	A tensor is an element of a tensor product.
\end{definition}

\begin{example}
	Tensor products of sections of tangent bundles and cotangent bundles are also the operations of differentiations of tensors.

	Recall that tangent bundles are bundles with a total object and a morphism to the base object of tangent space. Dualising this morphism gives the definition of a cotangent bundle.
\end{example}

\begin{example}
	The tensor product of vector bundles is the vector bundle whose fiber over any point is the tensor product of modules of the respective fibres.
\end{example}

\begin{example}
	The smash product is the tensor product in the closed monoidal category of pointed sets.

	It is the quotient set of the cartesian product where all points with a base point are identified with a quotient to be equivalent.

	It can also be defined as the pushout of pushouts and tensor products formed in the category for two pointed categories as the category of pointed objects (coslice category with maps FROM the terminal object) the smash product makes the coslice category also a closed symmetric monoidal category with finite limits and colimits.
\end{example}

\begin{definition}
	The cartesian product of object is in a cartesian category is such that it is a product object that is the product of objects equipped with projection morphisms to its component objects with the universal property that it factors uniquely through this cartesian product and into its respective projection morphisms. A cartesian category admits all finite limits.
\end{definition}

\begin{example}
	A cartesian closed product is a category with finite product closed under its cartesian monoidal structure.

	A closed category has hom-objects. Therefore, like the real numbers, the internal hom in a cartesian closed category is rightfully called exponentiation.

	A cartesian closed functor is a functor that preserves products as limits and preserves exponentiation as internal homs.
\end{example}

\begin{example}
	The Tor functor is the derived tensor product. It is the left derived functor for the tensor product of modules over a commutative ring.
\end{example}

\end{document}