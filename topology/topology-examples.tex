\section{Examples}

\begin{example}[Euclidean metric space]
    \label{example-euclidean-metric}
    Work with one dimension.
    Suppose $X$ is the real line $\mathbb{R}^1$.
    One can put the Euclidean metric $\mu$ on it.
    For two points $x_1$ and $x_2$ in $X$, 
    and without loss of generality we use the fact that
    the reals are well ordered where $x_1 > x_2$:

    \begin{equation}
        \mu(x_1, x_2) = \sqrt{x_1^2 - x_2^2}
    \end{equation}

    The use of the ordering is not ideal.
    We can actually take the square of the metric $\mu$
    instead.
\end{example}

\index{topology!indiscrete}
\begin{example}[Indiscrete topology]
    \label{example-indiscrete}
    Let $X$ be a set.
    The topology consisting of the empty set and the set $X$ is the indiscrete topology.
\end{example}

\begin{proof}
    The arbitrary union of all open subsets is the open set which is the space $X$.
    The finite intersection of disjoint open subsets is the empty set.
\end{proof}

\index{topology!indiscrete}
\begin{remark}[Indiscrete topology remarks]
    \label{remarks-indiscrete}
    \begin{enumerate}
        \item This topology is comparable to every other topology.
    \end{enumerate}
\end{remark}

\begin{proof}
    Every topology $\tau$ must contain the space $X$ and the empty set. This is because of the following reasons:

    The arbitrary union of all open subsets is the open set which is the space $X$.
    The finite intersection of disjoint open subsets is the empty set.

    Therefore, the indiscrete topology is comparable to every other topology.
\end{proof}

\index{topology!discrete}
\begin{example}[Discrete topology]
    \label{example-discrete}
    Let $X$ be a set.
    The topology consisting of all subsets of the set $X$ is the discrete topology.
\end{example}

\begin{proof}
    Arbitrary unions of subsets of a set $X$ is a subset of the set $X$ (recall that the full set $X$ is a subset of the set $X$).

    Finite intersections of subsets of a set $X$ is a subset of the set $X$.
\end{proof}

\index{topology!finite particular point}
\index{set!finite}
\begin{example}[Finite particular point topology]
    \label{example-finite-particular-point}
    Let $X$ be a finite set.
    Let $p$ be a point in the finite set $X$.
    Consider the topology $\tau$ consisting of arbitrary unions and finite intersection of subsets of the finite set containing the point $p$.
\end{example}

\begin{proof}
    This is a topology by definition.
    We make such a topology $\tau$ coarser by including only the open sets in the collection $\tau$ that contain the point $p$.
\end{proof}

\index{topology!countably infinite particular point}
\index{set!countably infinite}
\begin{example}[Countably infinite particular point topology]
    \label{example-countably-infinite-particular-point}
    Let $X$ be a countably inffinite set.
    Let $p$ be a point in the countably infinite set $X$.
    Consider the topology $\tau$ consisting of arbitrary unions and finite intersection of subsets of the finite set containing the point $p$.
\end{example}

\begin{proof}
    This is a topology by definition.
    We make such a topology $\tau$ coarser by including only the open sets in the collection $\tau$ that contain the point $p$.
\end{proof}

\index{topology!uncountably infinite particular point}
\index{set!uncountably infinite}
\begin{example}[Uncountably infinite particular point topology]
    \label{example-uncountably-infinite-particular-point}
    Let $X$ be a uncountably infinite set.
    Let $p$ be a point in the uncountably infinite set $X$.
    Consider the topology $\tau$ consisting of arbitrary unions and finite intersection of subsets of the finite set containing the point $p$.
\end{example}

\begin{proof}
    This is a topology by definition.
    We make such a topology $\tau$ coarser by including only the open sets in the collection $\tau$ that contain the point $p$.
\end{proof}

\index{topology!Zariski}
\begin{example}[Zariski topology]
    \label{example-zariski}
    Let $X$ be the collection of algebraic sets.
    An algebraic set is the locus of zeros of polynomials.
    The Zariski topology is the topology formed by the collection of algebraic sets of a space as the closed sets.\sidenote{This is not fully defined since we did not say what an algebraic set really is. And we need to prove this.}
\end{example}

\begin{proof}
    Omitted. To be filled in.
\end{proof}