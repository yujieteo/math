\section{Exercises}

\begin{exercise}
    \label{exercise-comparable-topology}
Show that not all topologies are comparable.
\end{exercise}

\begin{proof}
    Consider a set of two elements, with the finite particular point topology on each point as per Example \ref{example-finite-particular-point}. The topologies generated are not comparable.
\end{proof}

\begin{exercise}
    \label{exercise-open-closed}
Show that a subset in a topology can be open and closed.
\end{exercise}

\begin{proof}
    Consider an two point set with the indiscrete topology as per Example \ref{example-indiscrete}.
\end{proof}

\begin{exercise}
    \label{exercise-comparable}
Show that a topology can be comparable to every other topology.
\end{exercise}

\begin{proof}
    Consider a set with the indiscrete topology as per Example \ref{example-indiscrete}. It is coarser than every other topology.
    
    By definition, a topology must contain the indiscrete topology which consists of arbitrary unions of all subsets which is the full set, as well as finite intersections of all subsets which is the empty set for disjoint sets.
\end{proof}