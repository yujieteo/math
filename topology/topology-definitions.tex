\section{Definitions}

We follow Steen and Seebach's presentation of topology \cite{Steen1995}.

\index{topology}
\begin{definition}[Topology of open sets]
	\label{definition-topology-open-sets}
	For a set $X$, a collection of \textit{open} subsets 
	of the set $X$ is called a topology, denoted by $\tau$ 
	if arbitrary unions and finite intersections of each subset
	is in $\tau$.
\end{definition}

\index{topology}
\begin{definition}[Topology of closed sets]
	\label{definition-topology-closed-sets}
	For a set $X$, a collection of \textit{closed} subsets 
	of the set $X$ is called a topology, denoted by $\tau$ 
	if arbitrary intersections and finite unions of each subset
	is in $\tau$.
\end{definition}

\index{set!open}
\begin{definition}[Open set]
	\label{definition-open-set}
	An open set is a set $U$ in a topology $\tau$ of a set $X$.
\end{definition}

\index{set!closed}
\begin{definition}[Closed set]
	\label{definition-closed-set}
	An closed set $S$ is the complement of an open set $U$ 
	of a topology $\tau$ with respect to the main set $X$.
\end{definition}

\index{topology!closed sets}
\begin{lemma}[Topology of closed sets]
	The closed sets $(X - U)^\mathrm{c}$ 
	of a topological space $(X, \tau)$ form
	a topology $\tau$ given the open sets $U$ in the topology $\tau$.
\end{lemma}

\begin{proof}
	The complement of the entire space $X$ relative to finite intersection of open sets $U$
	is a arbitrary union of closed sets, this follows by De Morgan's laws or Lemma \ref{lemma-de-morgan-first} and Lemma \ref{lemma-de-morgan-second}.
	The complement of the entire space $X$ relative to arbitrary union of open sets $U$
	is a finite intersection of closed sets, this follows by De Morgan's laws.
\end{proof}

\index{topology}
\index{space!topological}
\begin{definition}[Topological space]
	\label{definition-topological-space}
	For a set $X$, a collection of subsets 
	of the set $X$ is called a topology, denoted by $\tau$ 
	if arbitrary unions and finite intersections of each subset
	is in $\tau$. A pair $(X, \tau)$ is a topological space.
	By abuse of notation, we call $X$ a topological space.
\end{definition}

\index{coarser}
\begin{definition}[Coarser]
	\label{definition-coarser}
	Suppose $\tau_1$ and $\tau_2$ are topologies for a set $X$.
	Recall that these are collections of subsets.\sidenote{Topologies may not be comparable.}
	
	If the set $\tau_1$ is contained in the set $\tau_2$, we say the topology 
	$\tau_1$ is coarser than $\tau_2$.
\end{definition}

\index{finer}
\begin{definition}[Finer]
	\label{definition-finer}
	Suppose $\tau_1$ and $\tau_2$ are topologies for a set $X$.
	Recall that these are collections of subsets.\sidenote{Topologies may not be comparable.}
	
	If the set $\tau_1$ is contained in the set $\tau_2$, we say the topology 
	$\tau_2$ is coarser than $\tau_1$.
\end{definition}

\index{neighbourhood}
\begin{definition}[Neighbourhood]
	\label{definition-neighbourhood}
	Suppose $\tau$ is a topology for a set $X$.
	Let $p$ be a point in the set $X$.
	A neighbourhood of a point $p$ in the set $X$ is a subset of an open set $U$ in the topology $\tau$ containing the point $p$.
\end{definition}

\index{neighbourhood}
\begin{definition}[Open neighbourhood]
	\label{definition-open-neighbourhood}
	Suppose $\tau$ is a topology for a set $X$.
	Let $p$ be a point in the set $X$.
	A open neighbourhood of a point $p$ in the set $X$ is a open subset of an open set $U$ in the topology $\tau$ containing the point $p$.
\end{definition}