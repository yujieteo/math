\section{Definitions}

We follow Steen and Seebach's presentation of topology \cite{Steen1995}.

\index{topology}
\begin{definition}[Topology of open sets]
	\label{definition-topology-open-sets}
	For a set $X$, a collection of \textit{open} subsets 
	of the set $X$ is called a topology, denoted by $\tau$ 
	if arbitrary unions and finite intersections of each subset
	is in $\tau$.
\end{definition}

\index{topology}
\begin{definition}[Topology of closed sets]
	\label{definition-topology-closed-sets}
	For a set $X$, a collection of \textit{closed} subsets 
	of the set $X$ is called a topology, denoted by $\tau$ 
	if arbitrary intersections and finite unions of each subset
	is in $\tau$.
\end{definition}

\index{set!open}
\begin{definition}[Open set]
	\label{definition-open-set}
	An open set is a set $U$ in a topology $\tau$ of a set $X$.
\end{definition}

\index{set!closed}
\begin{definition}[Closed set]
	\label{definition-closed-set}
	An closed set $S$ is the complement of an open set $U$ 
	of a topology $\tau$ with respect to the main set $X$.
\end{definition}

\index{topology!closed sets}
\begin{lemma}[Topology of closed sets]
	The closed sets $(X - U)^\mathrm{c}$ 
	of a topological space $(X, \tau)$ form
	a topology $\tau$ given the open sets $U$ in the topology $\tau$.
\end{lemma}

\begin{proof}
	The complement of the entire space $X$ relative to finite intersection of open sets $U$
	is a arbitrary union of closed sets, this follows by De Morgan's laws or Lemma \ref{lemma-de-morgan-first} and Lemma \ref{lemma-de-morgan-second}.
	The complement of the entire space $X$ relative to arbitrary union of open sets $U$
	is a finite intersection of closed sets, this follows by De Morgan's laws.
\end{proof}

\index{topology}
\index{space!topological}
\begin{definition}[Topological space]
	\label{definition-topological-space}
	For a set $X$, a collection of subsets 
	of the set $X$ is called a topology, denoted by $\tau$ 
	if arbitrary unions and finite intersections of each subset
	is in $\tau$. A pair $(X, \tau)$ is a topological space.
	By abuse of notation, we call $X$ a topological space.
\end{definition}

\begin{definition}[Open covers]
	\label{definition-open-cover}
	For a set $X$, a collection of subsets 
	of the set $X$ is called a topology, denoted by $\tau$ 
	if arbitrary unions and finite intersections of each subset
	is in $\tau$. A pair $(X, \tau)$ is a topological space.
	
	An open cover of $U$ on a open subset $V$
	of a a topological space $(X, \tau)$
	is a union of open sets $U = \cup U_i$ in 
	the topology $\tau$ such that 
	the open subset $V$ is contained in the open cover $U$.
\end{definition}

\begin{definition}[Finite open covers]
	\label{definition-finite-open-cover}
	For a set $X$, a collection of subsets 
	of the set $X$ is called a topology, denoted by $\tau$ 
	if arbitrary unions and finite intersections of each subset is in $\tau$. A pair $(X, \tau)$ is a topological space.
	
	A finite open cover of $U$ on a open subset $V$
	of a a topological space $(X, \tau)$
	is a finite union of open sets $U = \cup U_i$ in 
	the topology $\tau$ such that 
	the open subset $V$ is contained in the open cover $U$.
\end{definition}

\index{compact}
\begin{definition}[Compact topological space]
	\label{definition-compact-topological-space}
	For a set $X$, a collection of subsets 
	of the set $X$ is called a topology, denoted by $\tau$ 
	if arbitrary unions and finite intersections of each subset
	is in $\tau$. A pair $(X, \tau)$ is a topological space.
	
	A compact topological space $(X, \tau)$ is a topological space where every open cover, that is a union of a collection open sets $U = \cup U_i$ containing $X$
	admits a finite subcover, that is a finite union of 
	a subcollection of open sets $U = \cup U_i$.
\end{definition}

We now give the definition in terms of category theory.

\begin{remark}
	The category of topological spaces is both complete and cocomplete. From Definition \ref{definition-cocomplete} and \ref{definition-complete}, all small limits and colimits exist in the category of topological spaces.
\end{remark}

\index{compact}
\index{object!compact}
\begin{definition}[Compact topological space, compact object]
	\label{definition-compact-topological-space-cat}
	A compact topological space $(X, \tau)$ is a compact object in the category of topological spaces.\sidenote{See Definition \ref{definition-compact-object} for a general definition}

	A compact object is a corepresentable functor (hom object, in this case it is a topological space representing a set of open covers) from a locally small category\sidenote{We used the fact that the category of topological spaces is complete and cocomplete.} that admits filtered colimits such that homs out of it (open covers as homs) to a fixed object (the underlying topological space) preserve filtered colimits (finite subcovers).
\end{definition}

Colimits in the category of topological spaces are precisely the union of open covers. Filtered colimits in the category of topological spaces are precisely the union of finite covers. Admission means that there are finite subcovers.

\index{object!initial}
\begin{definition}[Initial object in the category of topological spaces]
	The unique initial object in the category of topological spaces is the empty set.
\end{definition}

\index{object!final}
\begin{definition}[Final object in the category of topological spaces]
	The final object in the category of topological spaces is the singleton space.
\end{definition}

\index{coarser}
\begin{definition}[Coarser]
	\label{definition-coarser}
	Suppose $\tau_1$ and $\tau_2$ are topologies for a set $X$.
	Recall that these are collections of subsets.\sidenote{Topologies may not be comparable.}
	
	If the set $\tau_1$ is contained in the set $\tau_2$, we say the topology 
	$\tau_1$ is coarser than $\tau_2$.
\end{definition}

\index{finer}
\begin{definition}[Finer]
	\label{definition-finer}
	Suppose $\tau_1$ and $\tau_2$ are topologies for a set $X$.
	Recall that these are collections of subsets.\sidenote{Topologies may not be comparable.}
	
	If the set $\tau_1$ is contained in the set $\tau_2$, we say the topology 
	$\tau_2$ is coarser than $\tau_1$.
\end{definition}

\index{neighbourhood}
\begin{definition}[Neighbourhood]
	\label{definition-neighbourhood}
	Suppose $\tau$ is a topology for a set $X$.
	Let $p$ be a point in the set $X$.
	A neighbourhood of a point $p$ in the set $X$ is a subset of an open set $U$ in the topology $\tau$ containing the point $p$.
\end{definition}

\index{neighbourhood}
\begin{definition}[Open neighbourhood]
	\label{definition-open-neighbourhood}
	Suppose $\tau$ is a topology for a set $X$.
	Let $p$ be a point in the set $X$.
	A open neighbourhood of a point $p$ in the set $X$ is a open subset of an open set $U$ in the topology $\tau$ containing the point $p$.
\end{definition}

\index{point!limit}
\begin{definition}[Limit point]
	\label{definition-limit-point}
	Suppose $\tau$ is a topology for a set $X$.
	Let $p$ be a point in the set $X$.
	A limit point $p$ in the set $X$ is a point such that it is in every open set contains $p$ and one distinct point that is not $p$.
	\sidenote{The motivation is that of a limit point of a sequence.}
\end{definition}

\index{point!adherent}
\begin{definition}[Adherent point]
	\label{definition-adherent-point}
	Suppose $\tau$ is a topology for a set $X$.
	Let $p$ be a point in the set $X$.
	A adherent point $p$ in the set $X$ is a point such that it is in every open set contains $p$ and one other point that may be equal to the point $p$.
\end{definition}

\index{point!$\omega$-accumulation}
\begin{definition}[$\omega$-accumulation point]
	\label{definition-omega-accumulation-point}
	Suppose $\tau$ is a topology for a set $X$.
	Let $p$ be a point in the set $X$.
	A $\omega$-accumulation point $p$ in the set $X$ is a point such that it is in every open set contains $p$ and infinitely many points that is not the point $p$.
\end{definition}

\index{point!condensation}
\begin{definition}[Condensation point]
	\label{definition-condensation-point}
	Suppose $\tau$ is a topology for a set $X$.
	Let $p$ be a point in the set $X$.
	A condensation point $p$ in the set $X$ is a point such that it is in every open set contains $p$ and uncountably infinitely many points that is not the point $p$.
\end{definition}

\index{set!derived}
\begin{definition}[Derived set]
	\label{definition-derived-set}
	Suppose $\tau$ is a topology for a set $X$.
	A derived set $D(A)$ of a set $A$ which is a subset of the set $X$ under the topology $\tau$ is the collection of all the limit points of the subset $A$.
\end{definition}

\index{point!isolated}
\begin{definition}[Isolated point]
	\label{definition-isolated-point}
	Suppose $\tau$ is a topology for a set $X$.
	A derived set $D(A)$ of a set $A$ which is a subset of the set $X$ under the topology $\tau$ is the collection of all the limit points of the subset $A$.

	An isolated point is a point in the subset $A$ that is not in the derived set $D(A)$.
\end{definition}

\index{set!dense in itself}
\begin{definition}[Dense in itself]
	\label{definition-dense-in-itself}
	Suppose $\tau$ is a topology for a set $X$.
	A derived set $D(A)$ of a set $A$ which is a subset of the set $X$ under the topology $\tau$ is the collection of all the limit points of the subset $A$.

	An isolated point is a point in the subset $A$ that is not in the derived set $D(A)$.

	A set without any isolated point is a set that is dense in itself.
\end{definition}

\index{set!perfect}
\begin{definition}[Perfect set]
	\label{definition-perfect}
	Suppose $\tau$ is a topology for a set $X$.
	A derived set $D(A)$ of a set $A$ which is a subset of the set $X$ under the topology $\tau$ is the collection of all the limit points of the subset $A$.

	An isolated point is a point in the subset $A$ that is not in the derived set $D(A)$.

	A closed set without any isolated point is a set that is dense in itself. This is defined to be a perfect set.
\end{definition}

\index{set!closure}
\begin{definition}[Closure set]
	\label{definition-closure}
	Suppose $\tau$ is a topology for a set $X$.
	The closure of a set is a set together with its limit points.
\end{definition}

\index{topological space!pointed}
\begin{definition}[Pointed topological space]
    \label{definition-pointed-topological-space}
    A pointed object is an object equipped with a global element.\sidenote{See Definition \ref{definition-pointed-object}.}

    Recall from Definition \ref{definition-global-element-final} that a global element is a morphism from the terminal object to that object.

    Therefore, a pointed topological space is a pointed object in the category of set. Explicitly, it is a topological set, equipped with a continuous map from a singleton in that topological space to the pointed topological space.
\end{definition}

\index{topological space!connected}
\begin{definition}[Connected topological space, connected objects]
    \label{definition-connected-topological-space}
    A connected object is an object whose hom functor out of the object to a fixed object is preserves coproducts.
    \sidenote{The colimit of connected objects is a connected object.}

	A connected topological space is a connected object in the category of topological spaces.
	This means that the map:
	\begin{equation}
		\mathrm{Hom}(X,Y) + \mathrm{Hom}(X,Z)
		\rightarrow \mathrm{Hom}(X,Y+Z)
	\end{equation}
	is a bijection on the level of sets.
\end{definition}