\documentclass[10pt]{article}

\usepackage{times}
\usepackage{amsthm}

\theoremstyle{plain}% default
\newtheorem{theorem}{Theorem}[section]
\newtheorem{lemma}[theorem]{Lemma}
\newtheorem{proposition}[theorem]{Proposition}
\newtheorem*{corollary}{Corollary}

\theoremstyle{definition}
\newtheorem{definition}{Definition}[section]
\newtheorem{conjecture}{Conjecture}[section]
\newtheorem{example}{Example}[section]
\newtheorem{exercise}{Exercise}[section]

\theoremstyle{remark}
\newtheorem*{remark}{Remark}
\newtheorem*{note}{Note}
\newtheorem{case}{Case}

\begin{document}

\title{Introducing representations with functors}

\maketitle

This presents a top down view of ordinary representation theory.

\begin{definition}
	A representation of a representable category in a representing category is a functor from the representable category to the representing category.
\end{definition}

\begin{definition}
	A homomorphism between representations is simply a natural transformation between representations.
\end{definition}

\begin{example}
	The representing category is the category of vector spaces over a field. The functor is a linear representation.
\end{example}

\begin{example}
	The representable category is the delooping of a group, it has a group representation in the representing category.
\end{example}

\begin{example}
	The representable category is the free category on a quiver. This is quiver representation.
\end{example}

\begin{example}
	The representable category is the category of $G$-sets for a group $G$. The representing category is a category of linear representation. The functor is a permutation representations.	
\end{example}


\end{document}