\documentclass[10pt]{article}

\usepackage{times}
\usepackage{amsthm}

\theoremstyle{plain}% default
\newtheorem{theorem}{Theorem}[section]
\newtheorem{lemma}[theorem]{Lemma}
\newtheorem{proposition}[theorem]{Proposition}
\newtheorem*{corollary}{Corollary}

\theoremstyle{definition}
\newtheorem{definition}{Definition}[section]
\newtheorem{conjecture}{Conjecture}[section]
\newtheorem{example}{Example}[section]
\newtheorem{exercise}{Exercise}[section]

\theoremstyle{remark}
\newtheorem*{remark}{Remark}
\newtheorem*{note}{Note}
\newtheorem{case}{Case}

\begin{document}

\title{Introducing representations with functors}

\maketitle

This presents a top down view of ordinary representation theory.

\begin{definition}
	A representation of a source category in a target category is a functor from the source category to the target category.
\end{definition}

\begin{definition}
	A homomorphism between representations is simply a natural transformation between representations.
\end{definition}

\begin{example}
	The target category is the category of vector spaces over a field. The functor is a linear representation.
\end{example}

\begin{example}
	The source category is the delooping of a group, it has a group representation in the target category.
\end{example}

\begin{example}
	The source category is the free category on a quiver. This is quiver representation.
\end{example}

\begin{example}
	The source category is the category of $G$-sets for a group $G$. The target category is a category of linear representation. The functor is a permutation representations.	
\end{example}

\begin{example}
	Let a category $C$ have a tensor product $\otimes$ and a unital tensor $I$ be a monoidal category. Let $|M|$ be the underlying object, multiplication $\mu : M \otimes M \rightarrow M$ and unit $\eta : I \rightarrow M$  satisfying the associative law and left and right unit laws such that we have a monoid in the monoidal category as a triple $(|M|, \mu, \eta)$.

	The regular representation is defined as the action of the monoid $M$ on the underlying object of $|M|$ in the category $C$ induced by the product $\mu$.
\end{example}

\begin{example}
	For a symmetric group $S_n$, its alternating representation is the one dimensional linear representation from the symmetric group $S_n$ to the general linear group of $GL(1)$ as a vector space sending even permutations to $+1$, and odd permutations to $-1$.
\end{example}

\begin{exercise}
	Define a symmetric group as a one object category with particular isomorphisms of relevance.

	Similarly, define the general linear group $GL(1)$.

	Now describe the alternating representation as a functor.
\end{exercise}

\begin{example}
	Let a Lie group be the target category (continuous symmetries of a single object where all morphisms are isomorphisms), and let the vector space underlying the Lie algebra given by the derivative of its adjoint action on the neutral element. This is the adjoint representation.
\end{example}

\begin{exercise}
	Define the co-adjoint action by defining a suitable vector space.
\end{exercise}

Recall the duality between subgroup inclusions and quotient groups. Likewise, we have the induction restriction adjunction in representation theory.

\begin{example}
	Every subgroup inclusion induces a restriction representation functor from the category of representations on the full group to the category of representations on the subgroup. This forgets the full group action on the given full representation, remembering only the subgroup action to induce the representation of the subgroups.

	If the restriction of a functor has a left adjoint, then there exists a functor assigning left induced representations from the category of representations on the subgroups back to the category of representations on the full group. 
	
	This is directly analogous to how the left adjoint of the restriction of scalars is the extension of scalars.
\end{example}

\begin{exercise}
	Use the example of the category of the finite dimensional representation over a field $k$ to construct the induced representation functor explicitly using the tensor product of representations.

	We have the induced representation functor from the vector space $V$ of the full representation to the tensor product $k[G] \otimes_H V$ of the $k$-algebra on the finite group $k[G]$ with the vector space $V$ under the tensor product in subgroup $H$.
	
	If you take $V = \mathbf{1}$, show that the induced representation is the basic permutation representation spanned by the coset space $G/H$:

	\begin{equation}
		\mathrm{ind}_H^{G} (\mathbf{1})
		= k[G / H]
	\end{equation}

\end{exercise}

\begin{example}
	Groups are symmetries of one object. Groupoids are symmetries of multiple objects.

	Therefore, we can define groupoid representations. Let $G$ be a groupoid. A linear representation of the groupoid $G$ is a groupoid homomorphism to the groupoid core $\mathrm{Core(Vect)}$ of the category of vector spaces $\mathrm{Vect}$. For each object in the groupoid there is an asscoiated vector space, and for each morphism in the groupoid a related linear map that respects composition and the identity.

	Picking the groupoid core of the category of sets give a permutation representation. We can write this symbolically using the power morphism notation $\mathrm{Core(Set)}^{G}$.

	Picking the groupoid to be the delooping groupoid of a group $G$ is a group representation.

	A convenient definition of monodromy is as follows. Let $X$ be a topological space. Forming monodromy is a functor from the category of covering spaces over $X$ to the permutation representations of the fundamental groupoid of $X$.
	\begin{equation}
		\mathrm{Fib_{E}} : \Pi_1(X) \rightarrow \mathrm{Set}
	\end{equation}

	This gives the fundamental theorem of covering spaces, where the reconstruction of covering spaces from monodromy is an inverse functor to the monodromy functor. Therefore, there is an equivalence of categories between the covering space $\mathrm{Cov(X)}$ with the permutation representations of the fundamental groupoid of $X$, denoted by $\mathrm{Set}^{\Pi_1(X)}$. This is also a form of duality.
\end{example}


\end{document}