\documentclass[10pt]{article}

\usepackage{times}
\usepackage{amsthm}

\theoremstyle{plain}% default
\newtheorem{theorem}{Theorem}[section]
\newtheorem{lemma}[theorem]{Lemma}
\newtheorem{proposition}[theorem]{Proposition}
\newtheorem*{corollary}{Corollary}

\theoremstyle{definition}
\newtheorem{definition}{Definition}[section]
\newtheorem{conjecture}{Conjecture}[section]
\newtheorem{example}{Example}[section]
\newtheorem{exercise}{Exercise}[section]

\theoremstyle{remark}
\newtheorem*{remark}{Remark}
\newtheorem*{note}{Note}
\newtheorem{case}{Case}

\begin{document}

\title{Bundles and their generalisations}

\maketitle

\begin{definition}
	A bundle $(E, p)_B$ over an object called the base object $B$ in a category $\mathbf{C}$ is an object called the total object $E$ of the category equipped with a morphism $p$ in the category from the total object $E$ to the base object $B$.
\end{definition}

The typical case is when these objects are spaces.

\begin{definition}
	Consider a generalised element $x$ in the base object $B$, the fibre of the total object over the generalised element $E_x$ of the bundle $(E, p)$ is the pullback $x*E$.
\end{definition}

By abuse of notation, we either truncate the morphism or truncate the total object for the statement of a bundle.

\begin{definition}
	Given two bundles $(E_1, p_1)_B$ and $(E_2, p_2)_B$, a morphism of bundles over the base object $B$ is a morphism of total objects $E_1$ to $E_2$ which commutes over the total object as a cocone. Alternatively, bundles over the base object $B$ form the slice category of category $\mathbf{C}$ over the base object $B$. 
\end{definition}

\begin{definition}
	The over category $\mathbf{C} / B$ is the cateogry of bundles over a given object.
\end{definition}

\begin{definition}
	A fibre bundle $(E, p)$ over base object $B$ with a standard fibre $F$ is a bundle over $B$ such that the fibre of the total object over the generalised element $E_x$ of the bundle $(E, p)$ is the pullback isomorphic to the standard fibre $x*E \cogr F$.
\end{definition}

\end{document}