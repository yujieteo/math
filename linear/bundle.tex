\documentclass[10pt]{article}

\usepackage{times}
\usepackage{amsthm}

\theoremstyle{plain}% default
\newtheorem{theorem}{Theorem}[section]
\newtheorem{lemma}[theorem]{Lemma}
\newtheorem{proposition}[theorem]{Proposition}
\newtheorem*{corollary}{Corollary}

\theoremstyle{definition}
\newtheorem{definition}{Definition}[section]
\newtheorem{conjecture}{Conjecture}[section]
\newtheorem{example}{Example}[section]
\newtheorem{exercise}{Exercise}[section]

\theoremstyle{remark}
\newtheorem*{remark}{Remark}
\newtheorem*{note}{Note}
\newtheorem{case}{Case}

\begin{document}

\title{Bundles and their generalisations}

\maketitle

\section{Bundles}

\begin{definition}
	A bundle $B^{(E,p)}$ over an object called the base object $B$ in a category $\mathbf{C}$ is an object called the total object $E$ of the category equipped with a morphism $p$ in the category from the total object $E$ to the base object $B$.
\end{definition}

The typical case is when these objects are spaces.

\begin{definition}
	Consider a generalised element $x$ in the base object $B$, the fibre of the total object over the generalised element $E_x$ of the bundle $B^{(E, p)}$ is the pullback $x^*E$.
\end{definition}

By abuse of notation, we either truncate the morphism or truncate the total object for the statement of a bundle.

\begin{definition}
	Given two bundles $B^{(E_1, p_1)}$ and $B^{(E_2, p_2)}$, a morphism of bundles over the base object $B$ is a morphism of total objects $E_1$ to $E_2$ which commutes over the total object as a cocone. Alternatively, bundles over the base object $B$ form the slice category of category $\mathbf{C}$ over the base object $B$. 
\end{definition}

\begin{definition}
	The over category $\mathbf{C} / B$ is the cateogry of bundles over a given object.
\end{definition}

\begin{definition}
	A fibre bundle $B^{(E, p)}$ over base object $B$ with a standard fibre $F$ is a bundle over $B$ such that the fibre of the total object over the generalised element $E_x$ of the bundle $B^{(E, p)}$ is the pullback (fibre product) isomorphic to the standard fibre $x^*E \cong F$.
\end{definition}

Coverage gives the minimum structure necessary to define a sheaf. Grothendieck weakened the notion of open covers (which gives the Zariski coverage)for more general coverage.

\begin{definition}
	A coverage of a category consists of each collections of covering families of morphisms on $U$ $f_i : U^{\{U_i\}_{i \in I}}$ for each object $U$ in the category such that if there is a morphism $g : U^V$ then there exist a covering family of morphisms $h_j = V^{\{V_j\}_{j \in J}}$
	such that there exist morphisms $k : V^{V_j}$ such that the composite of this morphism and the covering morphism on $V$, denoted as $g \circ h_j$ factors through the covering morphism on $U$, denoted as $f_i$.
\end{definition}

\begin{definition}
	A site is defined with the data of a category and its coverage.
\end{definition}

\begin{definition}
	A small site has a coverage with covering families that can be organised into a set.
\end{definition}

Note that the coverage is needed to be defined to exists.

\begin{definition}
	Suppose the category $C$ is a site.
	A locally trivial fibre bundle over base object $B$ is denoted as $B^{(E, p)}$ with standard fibre $F$ is a bundle over base object $B$ with a cover (covering family) $(j_\alpha : B^{U_\alpha})_\alpha$ such that for each index $\alpha$, the pullback $E_\alpha$ along $j_\alpha$ is isomorphic to the slice category $\mathbf{C} / U_\alpha$ between the pullback $E_\alpha$ and the trivial bundle $U_\alpha \times F_\alpha$.
\end{definition}

Every locally trivial fibre bundle is a locally trivial bundle. A locally trivial bundle is a locally trivial fibre bundle if the base object is connected.

\begin{definition}
	A cartesian category is a category whose monoid structure is given by the category theoretic product, with the terminal object as unit.
\end{definition}

\begin{definition}
	A group object or an internal group internal to a category with finite products (binary Cartesian products and a terminal object) is an object with unique morphisms to the terminal objects and morphisms within the category with a binary operation, a neutral element, and inverse elements such that diagrams that commute for a unital associative magma object also commute for the group object.
\end{definition}

\begin{definition}
	Suppose the category $\mathbf{C}$ is a site and a cartesian category.
	A $G$-bundle is a locally trivial fibre bundle over base object $B$ is denoted as $B^{(E, p)}$ with standard fibre $F$ is a bundle over base object $B$ with a cover (covering family or transition maps) $(g_{\alpha, \beta} : G^{U_{\alpha, \beta}})$ which give the transition maps relative to the action of the group object $G$ (a group internal to a cartesian category $\mathbf{C}$) on the standard fibre $F$.
\end{definition}

\begin{definition}
	A $G$-principal bundle is a $G$-bundle whose group object $G$ is the standard fibre.
\end{definition}

\begin{example}
	A $G$-torsor is a $G$-principal bundle over the base object $B$ in the category of $\mathbf{Set}$.

	If $G$ is a group object internal to $\mathbf{Set}$, then the action of the group $G$ on itself equips the underlying set of the group $G$ with a structure of a $G$-torsor.

	A unit of measurement is a torsor on the multiplicative group of the reals. An affine space over a base field is a torsor for the additive group of the base field by translation.
\end{example}

This means that for a $G$-principal bundle, the transition maps are automorphisms.

\begin{definition}
	Suppose $F$ is a topological vector space (over a topological field).
	
	A vector bundle is the case of a $\mathrm{GL}(F)$-bundle over $B$ with the standard fibre $F$, where $\mathrm{GL}(F)$ is the general linear group with its defining action on $F$.

	A module bundle is the case of a $\mathrm{GL}(F)$-bundle over $B$ with the standard fibre $F$ that is a topological module, where $\mathrm{GL}(F)$ is the general linear group with its defining action on $F$.
\end{definition}

\begin{example}
	A vector bundle with a base object a point is a vector space.

	A real vector bundle over a topological space is a real vector space associated with a $O(n)$-principal bundle. $O(n)$ is the orthogonal group, the group of rotations. Fibres are real vector spaces.

	Vector bundles can be dualised by passing fibrewise to dual vector space.

	A topological vector bundle has a base object being a topological space.

	A differentiable vector bundle has a base object a differentiable manifold.

	An algebraic vector bundle has a base object a scheme.

	A tangent bundle is a vector bundle with a base object a tangent space of a point.
	Application of direct sum and tensor product of vector space fibrewise gives equivalent definitions over vector bundles. Passing fibrewise to the cotangent space defines the cotangent bundle.
	
	The Grothendieck group completion under direct sum of vector bundles as monoid gives topological K-theory.
\end{example}

\begin{definition}
	A line bundle is a vector bundle of rank 1.
\end{definition}

\begin{exercise}
	Find a good definition for associated bundles.
\end{exercise}

\begin{example}
	Complex line bundles are canonically associated bundles of circle group prinicpal bundles.

	The Mobius strip is a unique up to isomorphism non trivial real line bundle over the circle group.
\end{example}

\section{Isbell adjunction on vector bundles}

With the Isbell adjunction between quantity and space, one can relate algebra with bundles.

\begin{enumerate}
	\item A ring can be thought of a ring of functions in a space.
	\item A module over a ring can be thought of a space of sections of a vector bundle on the space.
	\item A compact topological space can be thought of its C-star algebra of continuous function valued in the complex numbers. A opposite category of commutative C-star algebras is equivalent to the category of compact topological spaces. This is Gelfand duality.
	\item A Hausdorff topological complex vector bundle over a compact topological space can be thought of as a $C(X, \mathbf{C})$-module of its continuous sections. So topological complex vector bundles over a space $X$ is an equivalnce of categories to finitely generated projective modules over $C(X, \mathbf{C})$. This is the Serre-Swan theorem.
	\item The direct sum of modules is the fiberwise direct sum of vector bundles.
	\item The extension of scalars of a module along a ring homomorphism corresponds to a pullback of vector bundles along the dual map of spaces.
\end{enumerate}

\end{document}