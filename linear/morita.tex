\documentclass[10pt]{article}

\usepackage{times}
\usepackage{amsthm}
\usepackage{amssymb}

\theoremstyle{plain}% default
\newtheorem{theorem}{Theorem}[section]
\newtheorem{lemma}[theorem]{Lemma}
\newtheorem{proposition}[theorem]{Proposition}
\newtheorem*{corollary}{Corollary}

\theoremstyle{definition}
\newtheorem{definition}{Definition}[section]
\newtheorem{conjecture}{Conjecture}[section]
\newtheorem{example}{Example}[section]
\newtheorem{exercise}{Exercise}[section]

\theoremstyle{remark}
\newtheorem*{remark}{Remark}
\newtheorem*{note}{Note}
\newtheorem{case}{Case}

\begin{document}

\title{Morita theory as categorification of linear algebra}

\maketitle

\section{Commutative algebra}

I got interested in Morita theory because it helps generalise linear algebra. This is basically a copy of Qiaochu Yuan's blog post on Higher Linear Algebra. I have only included parts of it I am interested in.

\begin{definition}
	A unital ring is a monoid object in the category of an abelian groups. It is a monoid internal to the monoidal category of abelian groups. The monoidal category part is necessary to ensure that the tensor product of abelian groups exist.
\end{definition}

\begin{definition}
	A linear category or an algebroid is a category whose hom-sets are all modules with bilinear composition.

	A $k$-linear category or $k$-algebroid is a category enriched over the monoidal category of $k$-modules with the tensor product of modules.

	A symmetric monoidal $k$-linear category or $k$-algebroid is a category enriched over the category of $k$-modules with the tensor product of modules and the composition are bilinear. 
\end{definition}

\begin{example}
	A linear category is the horizontal categorification of (unital associative) algebra.
\end{example}

\begin{example}
	A integer algebra is a ring. A integer algebroid is a ringoid.
\end{example}

\begin{definition}
	A module category is a linear category equipped with an action of a tensor linear category. It is a category enriched in teh category of modules in a base field.
\end{definition}

A Morita 2-category can be thought of as the 2-category of module categories over the symmetric monoidal category $\mathrm{Mod}(k)$ of $k$-modules, equipped with the tensor product $\otimes_k$ over $k$.

\begin{corollary}[Corollary of Eilenberg-Watts]
	The Morita 2-category has objects as symmetric monoidal categories $\mathrm{Mod}(A)$ with $A$ as a $k$-algebra (a pair $(k, p)$ where $p$ is a $k$-ring homomorphism).

	The Morita 2-category has morphisms as co-continuous $k$-linear functors from symmetric monoidal categories of $k$-algebras $\mathrm{Mod}(A)$ to $\mathrm{Mod}(B)$.
\end{corollary}

Suppose $k$ is a commutative ring, and $\mathrm{Mod}(k)$ is a symmetric monoidal category with a tensor product $\otimes_k$ and a category $V$ as the base of enrichment.

\begin{definition}
	Consider categories enriched over the base of enrichment category $V$. These are the $V$-enriched categories.
\end{definition}

\begin{definition}
	Suppose $C$ and $D$ are two $V$-categories, the naive tensor product $C \otimes D$ is the $V$-category with objects are pairs of objects in $C$ and objects in $D$, and whose homs are given be the tensor product:

\begin{equation}
	\mathrm{Hom}_{C \otimes D} ((c_1, d_1), (c_2, d_2)) \cong \mathrm{Hom}_{C}(c_1, c_2) \otimes \mathrm{Hom}_{D}(d_1, d_2)
\end{equation}

\end{definition}

If we reduce these categories $C$ and $D$ to be one object categories, and the base of enrichment to be the $\mathrm{Mod}(k)$ which is a symmetric monoidal category with a tensor product $\otimes_k$, this reduces to the tensor product of $k$-algebras.

\begin{definition}
	Suppose $C$ and $D$ are two $V$-categories over the base of enrichment category $V$, a left module over $C$ is a $V$-enriched functor from $C$ to $V$. These can be thought of a co-presheaves.
\end{definition}

\begin{definition}
	Suppose $C$ and $D$ are two $V$-categories over the base of enrichment category $V$, a right module over $C$ is a $V$-enriched functor from the opposite category $C^\mathrm{op}$ to $V$.
	These can be thought of as presheaves.
\end{definition}

\begin{definition}
	Suppose $C$ and $D$ are two $V$-categories over the base of enrichment category $V$, a (C,D)-bimodule module is a $V$-enriched functor from the tensor product of categories $D^\mathrm{op} \otimes C$ to $V$. These exist since we assume the base of enrichment category $V$ is a monoidal category.
\end{definition}

These reduces to their usual meaning with one object categories to correspond to algebras, with the base of enrichment to be $\mathrm{Mod}(k)$ which is a symmetric monoidal category with a tensor product $\otimes_k$.

\section{Remarks on analogies}

\begin{enumerate}
	\item Sets are analogous to categories.
	\item Abelian groups are analogous to cocomplete categories. Abelian addition is categorified to be taking colimits. A cocomplete category has all small colimits. The presheaf category $[C^\mathrm{op}, \mathbf{Set}]$ is cocomplete for a small category $C$. The Yoneda embedding exhibits it as a free cocompletion of $C$ or $\mathrm{Yo} : C \hookrightarrow \mathrm{PSh}C$ as something that freely adjoins colimits to the small category $C$. 
	\item Rings are anaologous to monoidal cocomplete categories. The monoidal structure distributes bilinearly over colimits.
	\item Commutative rings are analogous to symmetric monoidal cocomplete categories.
	\item Modules over commutative rings are analogous to cocomplete module categories over symmetric monoidal cocomplete categories.
\end{enumerate}

By the universal property ofthe free cocompletion, cocontinuous (preserves all small colimits) $k$-linear functors from symmetric monoidal categories of $k$-algebras $\mathrm{Mod}(C)$ to $\mathrm{Mod}(D)$ correspond to $k$-linear functors from the category $C$ to the symmetric monoidal category of $k$-algebras $\mathrm{Mod}(D)$, or equivalently by adjunction to $(C,D)$-bimodules. Composition is the tensor product of bimodules, computed using co-ends. This proves and generalises the Eilenberg-Watts theorem.

We can have the category $C$ to be finitely many isomorphism classes of objects of objects. Replace this with the direct sum of one object from each isomorphism class because the category $C$ is Morita equivalent to the one object $k$-linear category with this endomorphism ring, now the Morita $2$-category is bigger since the category $C$ is allowed to have infinitely many objects.

\begin{definition}
	Suppose $C$ and $D$ are two $V$-categories over the base of enrichment category $V$

	The Morita 2-category has objects as essentially small $k$-linear categories $C$, morphisms as $(C,D)$-bimodules over $k$ and 2-morphisms as homomorphisms of bi-modules.

	Equivalently, objects are the cocomplete $k$-linear categories $\mathrm{Mod}(C)$ where these are have all small colimits, morphisms are cocontinuous $k$-linear functors from $\mathrm{Mod}(C)$ to $\mathrm{Mod}(D)$ preserving small colimits, and 2-morphisms as natural transformations.
\end{definition}

\begin{proposition}
	The Morita 2-category has biproduct.
	This is similar to the fact that vector spaces where coproducts and products coincide, making this a linear algebra.

	\begin{equation}
		\mathrm{Mod}(A) \times \mathrm{Mod}(B)
		\cong \mathrm{Mod}(A \sqcup B)
	\end{equation}
\end{proposition}

\begin{proposition}
	There is a tensor-hom adjunction
	\begin{equation}
		[\mathrm{Mod}(A) \otimes \mathrm{Mod}(B), \mathrm{Mod}(C)]
		\cong
		[\mathrm{Mod}(A), [\mathrm{Mod}(B), \mathrm{Mod}(C)]]
	\end{equation}
\end{proposition}

The internal hom is the category of cocontinuous $k$-linear functor which itself is a cocomplete $k$-linear category.

Using the definition of the universal property, we have:

\begin{equation}
	\mathrm{Mod}(A) \otimes \mathrm{Mod}(B) \cong \mathrm{Mod}(A \otimes_k B)
\end{equation}

where the naive tensor product is dentoed as $\otimes_k$ over the commutative ring $k$. Therefore, $\mathrm{Mod}_k$ is the unit object or the tensor product over $\mathrm{Mod}(k)$.

\section{Big categories and small categories}

Big categories are akin to categories of mathematical objects. These correspond to entier worlds like categories of sets, abelian groups, modules, sheaves. These tend to admit all small colimits (Cocomplete), and people consider cocontinuous left adjoints between them. Left adjoints preserve colimits.

Little categories are categories as mathematical objects. Categories with one objects are typical example, and serve as a starting point for enrichment, are Cauchy complete in the sense that they are the closure of a category under limits that are preserved for any functor.

We can pass from small to big categories by taking modules (most general) representations (left modules), or presheaves (right modules). 

We can pass from big to small categories by taking tiny objects.

\begin{definition}
	A tiny object is an object in the category such that the hom functor out of the object preserves colimits. These are also called small projective objects.
\end{definition}

\begin{example}
	A functor on a little category can be something like the category of open subsets of a topological subset, it might be a functor on a big category like the category of commutative rings.
\end{example}

\begin{example}
	A plausible definition for a cocomplete abelian category having a basis if it has tiny (compact and projective) generators.
\end{example}

Given a basis category $C$ for a higher module $\mathrm{Mod}(C)$, an object cn be described by a module / presheaf $F : C^\mathrm{op} \rightarrow \mathrm{Mod}(k)$.

The components of this presheaf can eb thought of as "coordinates" of the presheaf functor $\mathrm{F}$ with the category as the basis.

This is analogous to how a vector is the sum over a elements of a basis weighted by coordinates in that basis.

\begin{definition}
	A presheaf is a weighted colimit or coend or functor tensor product:

	\begin{equation}
		F(-) \cong \int^{c \in C}
		F(c) \otimes_k \mathrm{Hom}(-,c)
	\end{equation}

	weighted by coordinates $F(c)$ of the basis of representable presheaves.
\end{definition}

This is an enriched version of the co-Yoneda lemma where the presheaf of sets over a category is canonically a colimit of representable presheaves (integral).

Likewise, cocontinuous $k$-linear functor $\mathrm{Mod}(C)$ to $\mathrm{Mod}(C)$ are equivalent to bimodule functors $C \otimes_k D^\mathrm{op} \rightarrow \mathrm{Mod}(k)$ which can be interpreted as saying that such functors can be written as matrices indexed by categories $C$ and $D$. Composition and evaluations can be interepreted by familar linear algebra if we reinterpret the relevant products as tensor products, the relevant sums as co-ends.

One can degenerate this example by taking $C = D$. Then endormorphisms of $\mathrm{Mod}(C)$ correspond to $(C,C)$-bimodules or equivalently to functors $F : C^\mathrm{p} \otimes_k C \rightarrow \mathrm{Mod}(k)$. You can take the coends to define the trace of the endomorphism $F$ which is denoted by $\mathrm{Tr}(F)$ which in $\mathrm{Mod}(k)$.

\begin{equation}
	\mathrm{Tr}(F) = \int^{c \in C}
	F(c,c)
\end{equation}

This is a generalisation of the (zeroth) Hochschild homology, with coefficient in a bimodule, which it reduces to if $C$ is an algebra.

\begin{example}
	Generalise the Hochschild cohomology using a suitable computation using ends.
\end{example}

The identity functor an be represented by the hom $\mathrm{Hom(-,-)}$, taking its trace gets the Hochschild homology or the trace of $C$

\begin{equation}
	\mathrm{Tr}(C) = \int^{c \in C}
	\mathrm{Hom}(c,c)
\end{equation}

More explicitly, this coend is the result of coequalising the left and right action on the category $C$ on the hom $\mathrm{Hom(-,-)}$ by postcomposition and precomposition respectively. This is the relevant coequaliser:

\begin{equation}
	\mathrm{Tr}(C) = \mathrm{coeq}
	\left(
		\coprod_{c, d \in C} \mathrm{Hom}(c,d)
		\otimes
		\mathrm{Hom}(d,c)
		\rightrightarrows 
		\coprod_{c \in C}
		\mathrm{Hom}(c,c)
	\right)
\end{equation}

These send pairs of morphisms into mutual composites. This trace is the quotient of the direct sum of endomorphism rings of every object in the category $C$ by the subspace of 
"commutators" of the form $fg - gf$, where $f$ and $g$ need nor be endomorphisms.

\end{document}