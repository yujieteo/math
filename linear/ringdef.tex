
\documentclass[10pt]{article}

\usepackage{times}
\usepackage{amsthm}

\theoremstyle{plain}% default
\newtheorem{theorem}{Theorem}[section]
\newtheorem{lemma}[theorem]{Lemma}
\newtheorem{proposition}[theorem]{Proposition}
\newtheorem*{corollary}{Corollary}

\theoremstyle{definition}
\newtheorem{definition}{Definition}[section]
\newtheorem{conjecture}{Conjecture}[section]
\newtheorem{example}{Example}[section]
\newtheorem{exercise}{Exercise}[section]

\theoremstyle{remark}
\newtheorem*{remark}{Remark}
\newtheorem*{note}{Note}
\newtheorem{case}{Case}

\begin{document}

\title{Defining rings}

\maketitle

\begin{definition}
	A unital ring is a monoid object in the category of an abelian group. It is a monoid internal to the monoidal category of abelian groups. The monoidal category part is necessary to ensure that the tensor product of abelian groups exist.

	A unital ring is a pointed category enriched over the category of abelian groups with a single object.

	A unital ring is an abelian group equipped with a neutral element, a bilinear map or a group homomorphism out of the tensor product of abelian groups that is associative and unital.

	A ring object in the category of Sets is a ring, for Set is a cartesian monoidal category with a categorical theorectic product being the tensor product, and a neutral element being a terminal object.

	A commutative unital ring is a commutative monoid object in the monoidal category of abelian groups.
\end{definition}

\begin{example}
	One can change the abelian category of abelian groups with a higher category of symmetric monoidal higher groupoids, yielding ring groupoids or symmetric ring groupoids for the commutative case.
\end{example}

You want to drop unitality for analysis for some reason Borcherds explained but I don't understand. Commutativity is a nice plus, there are plenty of examples of noncommutative rings.

\end{document}