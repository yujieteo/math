\documentclass[10pt]{article}

\usepackage{times}
\usepackage{amsthm}

\theoremstyle{plain}% default
\newtheorem{theorem}{Theorem}[section]
\newtheorem{lemma}[theorem]{Lemma}
\newtheorem{proposition}[theorem]{Proposition}
\newtheorem*{corollary}{Corollary}

\theoremstyle{definition}
\newtheorem{definition}{Definition}[section]
\newtheorem{conjecture}{Conjecture}[section]
\newtheorem{example}{Example}[section]
\newtheorem{exercise}{Exercise}[section]

\theoremstyle{remark}
\newtheorem*{remark}{Remark}
\newtheorem*{note}{Note}
\newtheorem{case}{Case}

\begin{document}

\title{From categories to Kan extensions}

\maketitle

A category is about composition: it has (\textit{"collection"}) objects and a \textit{enriched base} of morphisms (organised as hom-objects under this \textit{enriched base}) with (\textit{single/multiple}) domain object(s) and (\textit{single/multiple}) codomain object(s) that are respected by composition and identities. Every object has an identity morphism. Morphisms compose with left and right units. Morphisms compose associatively.

% Hom-objects are used to remember that one can repeatedly generalise this to more and more objects.

Interpreting objects as "collections" generalises to internal categories. Having multiple domain or codomain objects generalises to operads. We neglect these choices for now. 

A functor $\mathcal{F}$ is a morphism from the category $\mathbf{C}$ to the category $\mathcal{F}\mathbf{C}$ respecting identification and composition. The morphisms in a functor category with objects as functors are natural transformations.

Kan extensions extends functors from other functors. Suppose a functor $\mathcal{F}$ is from a domain category $\mathbf{Dom}$ to a codomain category $\mathcal{F} \mathbf{Dom}$. A left non-pointwise Kan extension functor $\mathrm{Lan}_\mathcal{K} \mathcal{F}$ along an extension functor $\mathcal{K}$ from the extended domain category $\mathcal{K} \mathbf{Dom}$ to the codomain category $\mathcal{F} \mathbf{Dom}$ is such that the functor $\mathcal{F}$ has a unique natural transformation $\eta$ to the composition $\mathrm{Lan}_\mathcal{K} \mathcal{F} \circ \mathcal{K}$. The right non-pointwise Kan extension functor $\mathrm{Ran}_\mathcal{K} \mathcal{F}$ is similar with the direction of the natural transformation $\eta$ reversed.
This gives the natural isomorphism:
\begin{equation}
    [\mathbf{Dom}, \mathcal{F} \mathbf{Dom}](\mathcal{G} \mathcal{K}, \mathcal{F}) \cong 
    [\mathcal{K} \mathbf{Dom}, \mathcal{F} \mathbf{Dom}](\mathcal{G}, \mathrm{Ran}_\mathcal{K} \mathcal{F})
\end{equation}

A left Kan extension $(\mathcal{L}, \eta)$, and a right Kan extension $(\mathcal{R}, \epsilon)$ with unit $\eta$ from the identity functor $\mathbf{1}$ to $\mathcal{L} \mathcal{R}$ and counit $\epsilon$ from $\mathcal{L} \mathcal{R}$ to the identity functor $\mathcal{1}$ are such that the Kan extension functor $\mathcal{L}$ is left adjoint to right Kan extension functor $\mathcal{R}$.

Kan extensions along a uniqueness functor $\mathbf{!}$ can be used to define limits and colimits. A representable functor is a functor that can be represented by an object. A limit is an universal object is that replaces compositions of morphisms into it as a representing object that represents a diagram functor. Dually, a colimit subsumes compositions of morphisms out of it.

In practice, only pointwise Kan extensions are useful computationally. A pointwise Kan extension is characterised by preservation of limits and colimits with all representable functors. One uses a weighted limit (with respect to a weighting functor $\mathcal{W}$ from a indexing category $\mathcal{K} \mathbf{Dom}(f, \mathcal{K}-)$) for morphism $f$ in the codomain category $\mathcal{F} \mathbf{Dom}$:

\begin{equation}
    (\mathrm{Ran}_\mathcal{K} \mathcal{F})(f)
    :=
    \mathrm{lim}^{\mathcal{K} \mathbf{Dom}(f, \mathcal{K}-)} \mathcal{F}
\end{equation}

Dually,

\begin{equation}
    (\mathrm{Lan}_\mathcal{K} \mathcal{F})(f)
    :=
    \mathrm{colim}^{\mathcal{K} \mathbf{Dom}(\mathcal{K}-, f)} \mathcal{F}
\end{equation}

The presentation here is forced to be circular to conceptualise Kan extensions early. 

These notes are duality focused. Some dualities I think are important are: (1) Makkai duality, that corresponds logic to geometry in a categorical sense; (2) Tannaka duality, a duality that most generally establishes the Yoneda lemma; (3) Isbell duality, that corresponds algebra and geometry. A broad understanding of dualities and analogies sharpens thinking. This has great value outside of mathematics, since category theory is the mathematics of analogy.


\end{document}