%% filename: amsbook-template.tex
%% version: 1.1
%% date: 2014/07/24
%%
%% American Mathematical Society
%% Technical Support
%% Publications Technical Group
%% 201 Charles Street
%% Providence, RI 02904
%% USA
%% tel: (401) 455-4080
%%      (800) 321-4267 (USA and Canada only)
%% fax: (401) 331-3842
%% email: tech-support@ams.org
%% 
%% Copyright 2006, 2008-2010, 2014 American Mathematical Society.
%% 
%% This work may be distributed and/or modified under the
%% conditions of the LaTeX Project Public License, either version 1.3c
%% of this license or (at your option) any later version.
%% The latest version of this license is in
%%   http://www.latex-project.org/lppl.txt
%% and version 1.3c or later is part of all distributions of LaTeX
%% version 2005/12/01 or later.
%% 
%% This work has the LPPL maintenance status `maintained'.
%% 
%% The Current Maintainer of this work is the American Mathematical
%% Society.
%%
%% ====================================================================

%    AMS-LaTeX v.2 driver file template for use with amsbook
%
%    Remove any commented or uncommented macros you do not use.

\documentclass[10pt]{amsbook}

%    For use when working on individual chapters
%\includeonly{}

%    Include referenced packages here.
\usepackage{}

\newtheorem{theorem}{Theorem}[chapter]
\newtheorem{lemma}[theorem]{Lemma}

\theoremstyle{definition}
\newtheorem{definition}[theorem]{Definition}
\newtheorem{example}[theorem]{Example}
\newtheorem{xca}[theorem]{Exercise}

\theoremstyle{remark}
\newtheorem{remark}[theorem]{Remark}

\numberwithin{section}{chapter}
\numberwithin{equation}{chapter}

%    For a single index; for multiple indexes, see the manual
%    "Instructions for preparation of papers and monographs:
%    AMS-LaTeX" (instr-l.pdf in the AMS-LaTeX distribution).
\makeindex

\begin{document}

\frontmatter

\title{Notes on Category Theory}

%    Remove any unused author tags.

%    author one information
\author{Teo Yu Jie}
\address{}
\curraddr{}
\email{contact@teoyujie.org}
\thanks{}

%    author two information
\author{}
\address{}
\curraddr{}
\email{}
\thanks{}

\subjclass[2010]{Primary }

\keywords{}

\date{\today}

\begin{abstract}
\end{abstract}

\maketitle

%    Dedication.  If the dedication is longer than a line or two,
%    remove the centering instructions and the line break.
%\cleardoublepage
%\thispagestyle{empty}
%\vspace*{13.5pc}
%\begin{center}
%  Dedication text (use \\[2pt] for line break if necessary)
%\end{center}
%\cleardoublepage

%    Change page number to 6 if a dedication is present.
\setcounter{page}{4}

\tableofcontents

%    Include unnumbered chapters (preface, acknowledgments, etc.) here.
\include{Preface}

\mainmatter
%    Include main chapters here.

\chapter{Categories, functors, Kan extensions}

A category is about composition: it has (\textit{"collection"}) objects and a \textit{enriched base} of morphisms (organised as hom-objects under this \textit{enriched base}) with (\textit{single/multiple}) domain object(s) and (\textit{single/multiple}) codomain object(s) that are respected by composition and identities. Every object has an identity morphism. Morphisms compose with left and right units. Morphisms compose associatively.

% Hom-objects are used to remember that one can repeatedly generalise this to more and more objects.

Interpreting objects as "collections" generalises to internal categories. Having multiple domain or codomain objects generalises to operads. We neglect these choices for now. 

A functor $\mathcal{F}$ is a morphism from the category $\mathbf{C}$ to the category $\mathcal{F}\mathbf{C}$ respecting identification and composition. The morphisms in a functor category with objects as functors are natural transformations.

Kan extensions extends functors from other functors. Suppose a functor $\mathcal{F}$ is from a domain category $\mathbf{Dom}$ to a codomain category $\mathcal{F} \mathbf{Dom}$. A left non-pointwise Kan extension functor $\mathrm{Lan}_\mathcal{K} \mathcal{F}$ along an extension functor $\mathcal{K}$ from the extended domain category $\mathcal{K} \mathbf{Dom}$ to the codomain category $\mathcal{F} \mathbf{Dom}$ is such that the functor $\mathcal{F}$ has a unique natural transformation $\eta$ to the composition $\mathrm{Lan}_\mathcal{K} \mathcal{F} \circ \mathcal{K}$. The right non-pointwise Kan extension functor $\mathrm{Ran}_\mathcal{K} \mathcal{F}$ is similar with the direction of the natural transformation $\eta$ reversed.
This gives the natural isomorphism:
\begin{equation}
    [\mathbf{Dom}, \mathcal{F} \mathbf{Dom}](\mathcal{G} \mathcal{K}, \mathcal{F}) \cong 
    [\mathcal{K} \mathbf{Dom}, \mathcal{F} \mathbf{Dom}](\mathcal{G}, \mathrm{Ran}_\mathcal{K} \mathcal{F})
\end{equation}

A left Kan extension $(\mathcal{L}, \eta)$, and a right Kan extension $(\mathcal{R}, \epsilon)$ with unit $\eta$ from the identity functor $\mathbf{1}$ to $\mathcal{L} \mathcal{R}$ and counit $\epsilon$ from $\mathcal{L} \mathcal{R}$ to the identity functor $\mathcal{1}$ are such that the Kan extension functor $\mathcal{L}$ is left adjoint to right Kan extension functor $\mathcal{R}$.

Kan extensions along a uniqueness functor $\mathbf{!}$ can be used to define limits and colimits. A representable functor is a functor that can be represented by an object. A limit is an universal object is that replaces compositions of morphisms into it as a representing object that represents a diagram functor. Dually, a colimit subsumes compositions of morphisms out of it.

In practice, only pointwise Kan extensions are useful computationally. A pointwise Kan extension is characterised by preservation of limits and colimits with all representable functors. One uses a weighted limit (with respect to a weighting functor $\mathcal{W}$ from a indexing category $\mathcal{K} \mathbf{Dom}(f, \mathcal{K}-)$) for morphism $f$ in the codomain category $\mathcal{F} \mathbf{Dom}$:

\begin{equation}
    (\mathrm{Ran}_\mathcal{K} \mathcal{F})(f)
    :=
    \mathrm{lim}^{\mathcal{K} \mathbf{Dom}(f, \mathcal{K}-)} \mathcal{F}
\end{equation}

Dually,

\begin{equation}
    (\mathrm{Lan}_\mathcal{K} \mathcal{F})(f)
    :=
    \mathrm{colim}^{\mathcal{K} \mathbf{Dom}(\mathcal{K}-, f)} \mathcal{F}
\end{equation}

The presentation here is forced to be circular to show Kan extensions early to show this side of category theory early.

\appendix
%    Include appendix "chapters" here.
%    \include{}

\backmatter
%    Bibliography styles amsplain or harvard are also acceptable.
\bibliographystyle{amsalpha}
\bibliography{}
%    See note above about multiple indexes.
\printindex

\end{document}

%-----------------------------------------------------------------------
% End of amsbook-template.tex
%-----------------------------------------------------------------------
