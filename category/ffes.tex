\documentclass[10pt]{article}

\usepackage{times}
\usepackage{amsthm}

\theoremstyle{plain}% default
\newtheorem{theorem}{Theorem}[section]
\newtheorem{lemma}[theorem]{Lemma}
\newtheorem{proposition}[theorem]{Proposition}
\newtheorem*{corollary}{Corollary}

\theoremstyle{definition}
\newtheorem{definition}{Definition}[section]
\newtheorem{conjecture}{Conjecture}[section]
\newtheorem{example}{Example}[section]
\newtheorem{exercise}{Exercise}[section]

\theoremstyle{remark}
\newtheorem*{remark}{Remark}
\newtheorem*{note}{Note}
\newtheorem{case}{Case}

\begin{document}

\title{Full, faithful, essential surjective}

\maketitle

\begin{definition}
	A functor between small categories is a homomorphism of underlying graphs that respects the compositions of objects, consisting of a component function of classes of objects and sets of morphisms such that for each pair of objects, it has a component function between hom-sets respecting the source and target of morphisms with coincident source and target objects, respects the identity morphisms and respect composition of images.
\end{definition}

\begin{definition}
	A functor is full if for each pair of objects, the function between hom-sets is surjective.
\end{definition}

\begin{definition}
	A functor is faithful if for each pair of objects, the function between hom-sets is injective.
\end{definition}

\begin{definition}
	A functor is essentially surjective if there merely exists an isomorphism between the functor on a object in the source category to an object in the target category.
\end{definition}

Why the distinctions?

\begin{example}
	An essentially surjective functor is additionally fully faithful precisely when it is an equivalence of categories. This serves a good test for equivalence of categories.
\end{example}

What about infinity categories?

\begin{definition}
	A functor of infinity categories (in Lurie's construction, I copied this from Kerodon) is fully faithful if for every pair of objects in the infinity category, the induced map of morphism spaces is a homotopy equivalence of Kan complexes.
\end{definition}

\begin{exercise}
	Compare the definition of fully faithful to that in normal spaces to better understand what is important in infinity categories.
\end{exercise}

\end{document}