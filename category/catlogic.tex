\documentclass[10pt]{article}

\usepackage{times}
\usepackage{amsmath, amsthm, amssymb, amsfonts}

\theoremstyle{plain}% default
\newtheorem{theorem}{Theorem}[section]
\newtheorem{lemma}[theorem]{Lemma}
\newtheorem{proposition}[theorem]{Proposition}
\newtheorem*{corollary}{Corollary}

\theoremstyle{definition}
\newtheorem{definition}{Definition}[section]
\newtheorem{conjecture}{Conjecture}[section]
\newtheorem{example}{Example}[section]
\newtheorem{exercise}{Exercise}[section]

\theoremstyle{remark}
\newtheorem*{remark}{Remark}
\newtheorem*{note}{Note}
\newtheorem{case}{Case}

\begin{document}

\title{A bad summary of important ideas in categorical logic}

\maketitle

For proper definitions, see Lurie's notes. This is basically an even rougher version of this that can be totally wrong.

\begin{definition}
	A first order sentence can be built from finitely many sets from the relation symbol and basic logical operations and quantification over a set. It is also a formula with no free variables.

	A linearly ordered set satisfies reflexivity, transitivity, antisymmetry and a variant of the excluded middle in the form of first ordered sentences.
	
	A language consists a set of predicate symbols with arity. 
	 typed language additionally has a set of types such that the predicate symbols have arity which is a finite sequence of types.
	
	A first order theory is a language with a collection of first order symbols constructed using the symbols of the language. This collection is known as a first order theory.
\end{definition}

\begin{example}
	Pick the set of types to be empty, each predicate symbol is assigned true or false, this gives propositional logic.
\end{example}

The key definition is here, this is not a proper definition:

\begin{definition}
	An language-structure on language $L$ called an $L$-structure, is a set with a subset $M[P_i] \subseteq M^{n_i}$ for each predicate symbol as the set of all tuples satisfying predicate $P_i$.
	
	A first order sentence can be defined as true relative to a $L$-structure $M$.

	A model of a first order theory $T$ is a $L$-structure $M$ such that for each axiom in $T$ the axiom is assessed to be true in the $L$-structure $M$.
\end{definition}

\begin{definition}
	A first order theory and corresponding sentences is Godel complete if proofs of sentences can be derived from the axioms is an equivalent condition to saying that the sentence is true in every model of the theory. We can form an equivalence class of models of theories where sentences are true.
\end{definition}

\begin{remark}
	Syntax means we think about proofs of theories.

	Semantics means we think about how sentences are true in models of theories.
\end{remark}

\begin{definition}
	A function between models is an elementary embedding if we can have an equivalent decomposition of first order formulas with several tuple elements into other first order formulas. For example, if we have a first order formula in $M$, we can map it to a decomposed first order formula in $M'$ in this elementary embedding.
\end{definition}

\begin{definition}
	Consider a subset of a model under a formula and a first order theory, the formula is Makkai definable if for elementary embedding is preseved under preimage, and for a collection of models with their corresponding ultrafilter and utlraproducts, one can recover the syntax of a theory from the semantics.

	For a first order theory and its category of corresponding models, if it is Makkai definable, we say we can be recover up to equivalence from the structure of the category of corresponding models as a "category with ultraproducts" the corresponding theory. This defines Makkai's strong conceptual completeness or Makkai duality.

	More precisely, for a first order theory, the syntactic category can be recovered from the category of models under a suitable functor, there is a canonical equivalence between the syntactic category and the category of ultrafunctors from teh category of models to sets. An ultrafunctor is a functor compatible with ultraproducts in some sense.
	 
	The syntactic category is a example of a pretopos.
\end{definition}

\begin{example}
	Important features of the syntactic category are: it admits fibre products, forgetful functors to set preserve fiber products, and admits finite limits, carries effective epimorphisms (exhibit coequalisers on the level of maps projections of fibre products) to surjection of sets, and induced map for objects in the syntactic category as homomorphisms of upper semilattices, preserving least upper bounds of finite subsets.
\end{example}

The definition ofa coherent category is important, this makes some categories to syntactic categories.

\begin{definition}
	A category is coherent if it admits finite limits, admits factorisations of effective epimorphism and monomorphisms, admits some upper semilattice structrue on subobject, effective epimorphisms that are stable under pullback, and induced morphisms on the level of map of posets to have be a homomorphism of upper semilattices.

	A morphism of coherent categories is left exact functor, preserves effective epimorphisms, and preserves smallest elements and joins on the level of induced maps
	
	A model of a category is a morphism of coherent categories from the category to the category of sets.

	The category of models of a category is a full subcategory of the functor category from the coherent category to the category of set as spanned by the models of the category.
\end{definition}

The category of sets is a coherent category, this makes sense, since we can essentially look at the poset version of this, and restate the axioms for a coherent category in terms of order theory, see Lurie's lecture 4. A coherent category is a fancier version of a distributive lattice.

\begin{definition}
	A small Boolean category is a category such that there is an equivalence of categories between the category $\mathbf{C}$ itself and the weak syntactic category of a typed first order theory of the category itself
	$\mathrm{Syn}_0(T(\mathbf{C}))$.
	
	When this construction is possible, we call this construction the Booleanisation of the category.
\end{definition}

\begin{definition}
	The weak syntactic category of a typed first order theory can be pretopos completed to a full syntactic category.
\end{definition}

\begin{example}
	Some examples would include the construction of a spectral space to a Stone space.
\end{example}

\begin{definition}
	A model is consistent if it satisfies some form of Deligne completeness. The simplest example is that if a small coherent category is consistent, then there exists a model of the category. 

	A model of a category is a morphism of coherent categories from the category to the category of sets.
\end{definition}

\begin{definition}
	A pretopos is a category admitting finite elements, with all equivalence relations effective, admitting finite coproducts and coproducts are disjoint, with effective epimorphisms closed under pullbacks, and formation of finite coproducts preserved by pullback.
\end{definition}

\begin{definition}
	A Grothendieck topology on a category admitting finite limits specifies a family of coverings such that projection maps are coverings, whose fibre products are coverings, and becomes covering if some map that admits a section is also a covering.

	A Grothendieck site is a category with a Grothendieck topology on the category.
\end{definition}

\end{document}