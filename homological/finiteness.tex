\documentclass[10pt]{article}

\usepackage{times}
\usepackage{amsthm}

\theoremstyle{plain}% default
\newtheorem{theorem}{Theorem}[section]
\newtheorem{lemma}[theorem]{Lemma}
\newtheorem{proposition}[theorem]{Proposition}
\newtheorem*{corollary}{Corollary}

\theoremstyle{definition}
\newtheorem{definition}{Definition}[section]
\newtheorem{conjecture}{Conjecture}[section]
\newtheorem{example}{Example}[section]
\newtheorem{exercise}{Exercise}[section]

\theoremstyle{remark}
\newtheorem*{remark}{Remark}
\newtheorem*{note}{Note}
\newtheorem{case}{Case}

\begin{document}

\title{Introduction to finiteness conditions}

\maketitle

Consider a unital associative commutative ring $\mathrm{Mod}_A$ as $A$-modules. 

A module
$M$ in $\mathrm{Mod}_A$ is finitely generated 
if and only if there exists an $A$-linear
surjection $A^{\oplus n} \rightarrow M$
for nonzero $n$. The intuition for this is that there is a finite 
generating set (not a basis, since need not be linearly independent).

A module $M$ in 
$\mathrm{Mod}_A$ is finitely presented
if and only if there exists an exact sequence
$A^{\oplus m} \rightarrow A^{\oplus n} \rightarrow M \rightarrow 0$
for nonzero $m$ and $n$. The motivation is that categorically, 
these are the compact objects in the category of $A$-modules.
An example would be that finite sets are the compact objects in the category of sets.
If the commutative ring $A$ is a field, then the compact objects
are precisely the finite dimensional vector spaces.

\begin{lemma}
	Let $0 \rightarrow M' \rightarrow M \rightarrow M'' \rightarrow 0$
	be an exact sequence.
	
	If $M'$ and $M''$ are finitely generated, then so is $M$.
	
	If $M'$ and $M''$ are finitely presented, then so is $M$.
	
	If $M'$ is finitely generated and $M$ is finitely presented, then $M''$ is finitely presented.
	
	If $M$ is finitely generated then $M''$ is finitely generated.
	
	If $M$ is finitely generated, and $M''$ is finitely presented, then $M'$ is finitely generated.
\end{lemma}

\begin{lemma}
	If $A$ is noetherian, then $M$ is a finitely generated $A$-module if and only if $M$ is also a finitely presented $A$-module.
\end{lemma}

\begin{proof}
	Suppose $A$-module $M$ is finitely generated. 
	Consider $A^{\oplus n} \rightarrow M$. 
	Since the $A$-module $M$ is finitely generated, we 
	can choose $K$ to be such that
	$0 \rightarrow K \rightarrow A^{\oplus n} \rightarrow M \rightarrow 0$. 
	This is the key step: since $A$ is noetherian, therefore $K$ is finitely generated. 
	Since $K$ is also finitely generated, by definition one can construct 
	$A^{\oplus m} \rightarrow K \rightarrow A^{\oplus n}$.
	This gives the construction of an exact sequence 
	for $M$ to be finitely presented 
	where $A^{\oplus m} \rightarrow A^{\oplus n} \rightarrow M \rightarrow 0$.
	Therefore, the $A$-module $M$ must be finitely presented since this 
	construction satisfies the definition.
\end{proof}

\begin{example}
	An example would be an ideal. Let $I$ be in an ideal
contained in module $A$. Consider the surjection from $A$ to the 
quotient of $A$ over the ideal $A/I$. 
This quotient $A/I$ must be finitely generated.
\end{example}


\end{document}