\section{Definitions}

\subsection{Zermelo-Fraenkel Axioms}

This follows Jech \cite{Jech2006}.

\index{axiom!pairing}
\begin{definition}[Pairing axiom]
	\label{axiom-pairing}
	For any two sets $X$ and $Y$, then there exists a set, denote $\{X, Y\}$, where the set contains exactly $\{X, Y\}$.
\end{definition}

\index{axiom!extensionality}
\begin{definition}[Extensionality axiom]
	\label{axiom-extensionality}
	If sets $X$ and $Y$ have the same elements, we define equality where the set $X$ is equal to the set $Y$.
\end{definition}

\index{axiom!union}
\begin{definition}[Union axiom]
	\label{axiom-union}
	The union over elements of a set exists.
\end{definition}

\index{axiom!infinity}
\begin{definition}[Infinity axiom]
	\label{axiom-infinity}
	An infinite set exists.
\end{definition}

\index{axiom!regularity}
\begin{definition}[Regularity axiom]
	\label{axiom-regularity}
	All nonempty sets have a membership minimal element.
\end{definition}

\index{axiom schema!separation}
\begin{definition}[Separation axiom schema]
	\label{axiom-schema-separation}
	If $P$ is a property parameterised by $p$, then for any set $X$ and parmaeter $p$, then there exists a set $Y$ that has elements $y$ in $X$ that contains all elements $y$ in $X$ that has property $P$.
\end{definition}

\index{axiom!powerset}
\begin{definition}[Powerset axiom]
	\label{axiom-powerset}
	For any set $X$, there exists the set of all subsets of $X$ called the power set of $X$, and is denoted by $P(X)$.
\end{definition}

\index{axiom schema!replacement}
\begin{definition}[Replacement axiom schema]
	\label{axiom-schema-replacement}
	If a class $F$ is a function, there for any set $X$, there exists a set called the function set with elements of the form $F(x)$ for an element x in set $X$, this set is denoted $F(X)$.
\end{definition}

\index{axiom!strong choice}
\begin{definition}[Strong choice axiom]
	\label{axiom-strong-choice}
	All families of nonempty sets have a choice function.
\end{definition}
\subsection{Naive sets}

This section is purely metamathematical and it is not rigorous.

\index{set!naive}
\begin{definition}[Naive set]
	A naive set is a list of unique elements.
\end{definition}

\index{membership!naive}
\begin{definition}[Naive membership]
	If an element $x$ in in a naive set $X$, we say $x$ is a member in $X$, and it is denoted by $x \in X$.
\end{definition}

\index{set!union}
\begin{definition}[Union of sets]
	Let $X$ and $Y$ be naive sets.
	Then, there exists a set, called the union of $X$ and $Y$, denoted $X \cup Y$, which contains the list of unique elements in either $X$ or $Y$, or both.
\end{definition}

\index{set!union}
\begin{lemma}[Union exists]
	\label{lemma-union-exists}
	The union of naive sets exists.
\end{lemma}

\begin{proof}
	Use Axiom \ref{axiom-union} on all the elements in a pair of naive sets $X$ and $Y$. Use Axiom \ref{axiom-infinity} for infinite sets.
\end{proof}

\index{set!intersection}
\begin{definition}[Intersection of sets]
	Let $X$ and $Y$ be naive sets.
	Then, there exists a set, called the intersection of $X$ and $Y$, denoted $X \cap Y$, which contains the list of unique elements in $X$ and in $Y$. 
\end{definition}

\index{set!intersection}
\begin{lemma}[Intersection exists]
	\label{lemma-intersection-exists}
	The intersection of naive sets exists.
\end{lemma}

\begin{proof}
	Use Axiom \ref{axiom-union} on all the elements in a pair of naive sets $X$ and $Y$ in common. Use Axiom \ref{axiom-infinity} for infinite sets.
\end{proof}

\index{set!complementation}
\begin{definition}[Complementation of sets]
	Let $X$ and $Y$ be naive sets.
	Then, there exists a set, called the complementation of $X$ and $Y$, denoted $X - Y$, which contains the list of unique elements in $X$ but not in $Y$. 
\end{definition}

\index{set!complementation}
\begin{lemma}[Complementation exists]
	\label{lemma-complementation-exists}
	The complementation of naive sets exists.
\end{lemma}

\begin{proof}
	Use Axiom \ref{axiom-union} on all the elements in a pair of naive sets $X$ and $Y$ in common only considering elements in $X$ but exclude elements that are in both $X$ and $Y$. Use Axiom \ref{axiom-infinity} for infinite sets.
\end{proof}